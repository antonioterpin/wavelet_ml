\subsection{Proposed architecture}
\begin{frame}[fragile]{Proposed architecture}
    \vskip -0.5cm
    \begin{block}{Region-based CNN (R-CNN)}
        \begin{itemize}
            \item Region proposals
            \item Classification
        \end{itemize}
    \end{block}
    \vskip 0.25cm
    \centering
    \begin{tikzpicture}
        % Input
        \node[rectangle, draw, minimum width=2cm, minimum height=2cm, anchor=west] at (-2,0) {Image};

        \draw[->] (0.5,0) -- (1.5,0);

        % Region proposals
        \node at (2.5,1.5) {Region proposals};
        \foreach \i in {0,...,7}
            \node[rectangle, draw, minimum width=1cm, minimum height=1cm, anchor=north west, fill=white] at (2 +\i*0.2,1-\i*0.2) {};

        \draw[->] (4.5,0) -- (5.5,0);

        % Classification
        \node at (5.5,1.5) {Class};
        \draw[->, color=UniBlue] (6.2, .8) -- (5.7, 1.25);
        \draw[color=UniBlue] (6.35,-.25) ellipse (0.25cm and 1.1cm);

        \node at (8.2,2) {Encoded};
        \node at (8.3,1.5) {Pixels};
        \draw[->, color=UniOrange] (7, .95) -- (7.5, 1.55);
        \draw[color=UniOrange] (7.6,.6) ellipse (1.1cm and 0.35cm);

        \node [matrix, anchor=north west, draw] at (6,1)
        {
            % \node {Class}; & \node {EncodedPixels};\\
            \node {$1$}; & \node {``12 13 \ldots''};\\
            \node {$2$}; & \node {``6 35 \ldots''};\\
            \node {$3$}; & \node {``''};\\
            \node {$4$}; & \node {``192 78 \ldots''};\\
        };
    \end{tikzpicture}
\end{frame}

\subsubsection{Region proposals}
\begin{frame}{Region proposals}
    TODO include original image
    TODO include edged image
    TODO include alpha-shape
    TODO include bounding boxes
\end{frame}

\begin{frame}
    \frametitle{Region proposals}
    \framesubtitle{Optimization}
    \begin{description}
        \item<1-> 
        \begin{alertblock}{Alpha shape tuning}
            Manually tuning the parameters to satisfy the validity of the topologically correct image segmentation using alpha-shapes requires a full knowledge of the object considered.
        \end{alertblock}
        \item<2->
        \begin{exampleblock}{Alpha shape tuning - solution}
            An empirical approach is proposed instead. To tune the hyper parameters of the algorithm, bayesian optimization is used.
        \end{exampleblock}
    \end{description}
\end{frame}

\begin{frame}
    \frametitle{Region proposals}
    \framesubtitle{Optimization}
    \begin{description}
        \item<1->
        \begin{block}{Bayesian optimization}
            Bayesian optimization aims to solve an optimization problem $\max\limits_{x\in A}f\left(x\right)$ when:
            \begin{itemize}
                \item $A \subset \mathbb{R}^d$ with $d$ not too large.
                \item Easy to asses $A$ membership.
                \item $f$ can be modeled using \emph{Gaussian process regression}.
                \item $f$ is a \emph{black box}.
            \end{itemize}
        \end{block}
        \item<2->
        \begin{exampleblock}{Loss function}
            \begin{equation}
                J\left(\alpha\right) = \frac{2\lvert X \cap Y \rvert}{\lvert X \rvert + \lvert Y \rvert} = -f(x)
            \end{equation}
        \end{exampleblock}
    \end{description}
\end{frame}

\begin{frame}
    \frametitle{Region proposals}
    \framesubtitle{Optimization}
    \vskip -0.5cm
    \begin{block}{Bayesian optimization}
        Firstly a collection of points $x_1, \ldots, x_k \in \mathbb{R}^d$ is considered. Suppose $f$ can be modeled as:
        \begin{equation*}
            f\left(x_{1:n}\right) \sim \mathcal{N}\left(\mu\left(x_{1:n}\right), \Sigma\left(x_{1:n}, x_{1:n}\right)\right)
        \end{equation*}
        Then we can infere the probability distribution of $f\left(x_{n+1}\right)$:
        \begin{equation}
            f\left(x_{n+1}\right) \mid f\left(x_{1:n}\right) \sim \mathcal{N}\left(\mu_{n+1}, \Sigma_{n+1}\right)
        \end{equation}
        \begin{equation*}
            \mu_{n+1} = \Sigma\left(x_{n+1}, x_{1:n}\right)\Sigma\left(x_{1:n}, x_{1:n}\right)^{-1}\left(f\left(x_{1:n}\right) - \mu\left(x_{1:n}\right)\right) + \mu\left(x_{n}\right)
        \end{equation*}
        \begin{equation*}
            \Sigma_n = \Sigma\left(x_{n+1}, x_{n+1}\right) - \Sigma\left(x_{n+1}, x_{1:n}\right)\Sigma\left(x_{1:n}, x_{1:n}\right)^{-1}\Sigma\left(x_{1:n}, x_{n+1}\right)
        \end{equation*}
    \end{block}
\end{frame}

\begin{frame}
    \frametitle{Region proposals}
    \framesubtitle{Optimization}
    \begin{exampleblock}{Acquisition function}
        Therefore, it is possible to consider an \emph{acquisition function}, i.e. a function that given the previous $x_1, \ldots, x_n$ values determines the best choice of $x_{n+1}$:
        \begin{equation}
            a \colon \left[x_1 \ldots x_n\right] \rightarrow x_{n+1} \colon \mathcal{P}\left[f(x_{n+1}) = \max\limits_{x \in A} f\left(x\right)\right] \text{is max}
        \end{equation}
    \end{exampleblock}
\end{frame}

\begin{frame}
    \frametitle{Region proposals}
    \framesubtitle{Optimization}
    TABLE with results\\
    plot bayesian \\
    example results
\end{frame}

\subsubsection{MC-CNN}
\begin{frame}{Multi Column CNN (MC-CNN)}
    \centering
    \begin{tikzpicture}
        % image
        \node[circle, draw] (input) at (0,0) {I};

        % preprocessed image
        \node[rectangle, draw] (p1) at ($(input) + (2,2)$) {P};
        \node[rectangle, draw] (p2) at ($(input) + (2,.5)$) {P};
        \node (p3) at ($(input) + (2,-.5)$) {\vdots};
        \node[rectangle, draw] (p4) at ($(input) + (2,-2)$) {P};

        % image to preprocessed image
        \draw[->] ($(input) + (0.5, 0)$) -- ($(p1) + (-0.3, 0)$);
        \draw[->] ($(input) + (0.5, 0)$) -- ($(p2) + (-0.3, 0)$);
        \draw[->] ($(input) + (0.5, 0)$) -- ($(p4) + (-0.3, 0)$);

        % cnn
        \node[rounded rectangle, draw, minimum width=2.5cm, minimum height=1cm] (cnn1) at ($(p1) + (2.5,0)$) {CNN};
        \node[rounded rectangle, draw, minimum width=2.5cm, minimum height=1cm] (cnn2) at ($(p2) + (2.5,0)$) {CNN};
        \node[minimum width=2.5cm, minimum height=1cm] (cnn3) at ($(p3) + (2.5,0)$) {\vdots};
        \node[rounded rectangle, draw, minimum width=2.5cm, minimum height=1cm] (cnn4) at ($(p4) + (2.5,0)$) {CNN};

        % preprocessed to cnn
        \draw[->] ($(p1) + (0.5, 0)$) -- ($(cnn1) + (-1.2, 0)$);
        \draw[->] ($(p2) + (0.5, 0)$) -- ($(cnn2) + (-1.2, 0)$);
        \draw[->] ($(p4) + (0.5, 0)$) -- ($(cnn4) + (-1.2, 0)$);

        % classifier
        \node[rectangle, draw, minimum width=2cm, minimum height=1cm] (classifier) at ($(cnn3) + (3.5,0.5)$) {NN};

        % cnn to classifier
        \draw[->] ($(cnn1) + (1.3, 0)$) -- ($(classifier) + (-1.2, .2)$);
        \draw[->] ($(cnn2) + (1.3, 0)$) -- ($(classifier) + (-1.2, 0)$);
        \draw[->] ($(cnn4) + (1.3, 0)$) -- ($(classifier) + (-1.2, -.2)$);

        % output
        \node[circle, draw] (output) at ($(classifier) + (2,0)$) {O};

        % classifier to output
        \draw[->] ($(classifier) + (1.1, 0)$) -- ($(output) + (-.45, 0)$);

    \end{tikzpicture}
\end{frame}

\begin{frame}{Multi Column CNN (MC-CNN)}
    \includegraphics[width=\textwidth]{graphics/architecture/mc-cnn-input}
    $$\Downarrow$$
    \centering
    \begin{tabular}{|c|c|c|c|c|}
        \hline
        Flawless area & Defect \#1 & Defect \#2 & Defect \#3 & Defect \#4\\\hline
        $1\%$ & $4\%$ & $6\%$ & \cellcolor{UniBlue}\textcolor{white}{$87\%$} & $2\%$ \\
        \hline
    \end{tabular}
\end{frame}

\begin{frame}
    \frametitle{Multi Column CNN (MC-CNN)}
    \framesubtitle{Shape column}
    \includegraphics[width=\textwidth]{graphics/architecture/mc-cnn-shape}
    % \includegraphics[width=\textwidth]{graphics/architecture/mc-cnn-shape-architecture}
    TODO confusion matrix + learning curve
\end{frame}

\begin{frame}
    \frametitle{Multi Column CNN (MC-CNN)}
    \framesubtitle{Local column}
    \includegraphics[width=\textwidth]{graphics/architecture/mc-cnn-local}
    % \includegraphics[width=\textwidth]{graphics/architecture/mc-cnn-shape-architecture}
    TODO confusion matrix + learning curve
\end{frame}

\begin{frame}
    \frametitle{Multi Column CNN (MC-CNN)}
    \framesubtitle{Global column}
    \includegraphics[width=\textwidth]{graphics/architecture/mc-cnn-global}
    % \includegraphics[width=\textwidth]{graphics/architecture/mc-cnn-shape-architecture}
    TODO confusion matrix + learning curve
\end{frame}

\subsubsection{Output layer}
\begin{frame}{Output layer}
    \begin{tikzpicture}

        % columns
        \node[rounded rectangle, draw, minimum width = 2cm, minimum height=1cm] (cnn1) at (0,2) {Shape};
        \node[rounded rectangle, draw, minimum width = 2cm, minimum height=1cm] (cnn2) at (0,0) {Local};
        \node[rounded rectangle, draw, minimum width = 2cm, minimum height=1cm] (cnn3) at (0,-2) {Global};

        % columns output
        \foreach \i in {0,...,4}
            \node[rectangle, draw, minimum width=1cm, minimum height=1cm, anchor=north west, fill=white] (output1 \i) at ($(cnn1) + (2+\i*0.15, 1-\i*.15) $) {$87\%$};

        \foreach \i in {0,...,4}
            \node[rectangle, draw, minimum width=1cm, minimum height=1cm, anchor=north west, fill=white] (output2 \i) at ($(cnn2) + (2+\i*0.15, 1-\i*.15) $) {$92\%$};

        \foreach \i in {0,...,4}
            \node[rectangle, draw, minimum width=1cm, minimum height=1cm, anchor=north west, fill=white] (output3 \i) at ($(cnn3) + (2+\i*0.15, 1-\i*.15) $) {$42\%$};

        % columns to output
        \draw[->] ($(cnn1) + (1, 0)$) -- ($(cnn1) + (1.7, 0)$);
        \draw[->] ($(cnn2) + (1, 0)$) -- ($(cnn2) + (1.7, 0)$);
        \draw[->] ($(cnn3) + (1, 0)$) -- ($(cnn3) + (1.7, 0)$);

        % fuse output
        \node [matrix, anchor=west, draw] (finalinput) at ($(cnn2) + (4.5,0)$)
        {
            \node {$13\%$};\\
            \node {$25\%$};\\
            \node {\vdots};\\
            \node {$42\%$};\\
        };

        % cnn output to flatten
        \draw[->] ($(output1 4) + (.6, 0)$) -- ($(finalinput) + (-.7,.2)$);
        \draw[->] ($(output2 4) + (.6, 0)$) -- ($(finalinput) + (-.7,0)$);
        \draw[->] ($(output3 4) + (.6, 0)$) -- ($(finalinput) + (-.7,-.2)$);

        % hidden layer
        \foreach \i in {0,...,7}
            \node[circle, draw, minimum width=.75cm, minimum height=.75cm, anchor=west] (h \i) at ($(finalinput) + (2, 3-\i*.8) $) {};
        \foreach \i in {0,...,7}
            \draw[->] ($(finalinput) + (.75,0)$) -- ($(h \i) + (-.5,0)$);

        % output layer
        \foreach \i in {0,...,4}
            \node[circle, draw, minimum width=.75cm, minimum height=.75cm, anchor=west] (o \i) at ($(finalinput) + (4, 1.75-\i*.8) $) {};
        \foreach \i in {0,...,4}
            \foreach \j in {0,...,7}
                \draw[->] ($(h \j) + (.5,0)$) -- ($(o \i) + (-.5,0)$);
        
        
    \end{tikzpicture}
    % FC NN + bayesopt params....
\end{frame}