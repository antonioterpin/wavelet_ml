\section{Challenger}\label{section:challenger}
    \par{
        To measure the effectiveness of the R-CNN approach relying on the edge-based image segmentation involving wavelet analysis, the results are compared with another well-known architecture, i.e. a \emph{sliding window} classificator. This comparison is reported in \ref{section:results}.
    }
    \subsection{Sliding window architecture}
        \par{
            The idea behind this architecture is to crop the input image at different locations (eventually all the ones possible, as in this paper) and use a classifier to assign to the pixels of the considered region an array of confidences.
        }
        \begin{figure}
            \centering
            \begin{tikzpicture}
                % Image
                \node[rectangle, draw, minimum width=3cm, minimum height=3cm] (image) at (0,0) {Image};
                % Window
                \node[rectangle, draw=red, minimum width=.8cm, minimum height=.8cm] (window) at ($(image) + (.3,.9)$) {};
                % Sliding window
                \draw[->, draw=red] ($(window) + (.2,0)$) -- ($(window) + (1,0)$);
                \draw[->, draw=red] ($(window) + (0,-.2)$) -- ($(window) + (0,-.7)$);
                % Classifier
                \node[rounded rectangle, draw, minimum width=2cm, minimum height=1cm] (classifier) at ($(image) + (2.5,0)$) {Classifier};
                \draw[->,draw=blue] ($(window) + (0,0)$) -- (classifier);
                % Confidence map
                \node[rectangle,draw,minimum width=3cm, minimum height=3cm] (confidence map) at ($(classifier) + (2.5,0)$) {Confidence map};
                \node[rectangle, draw=red, minimum width=.8cm, minimum height=.8cm] (confidence map window) at ($(confidence map) + (.3,.9)$) {};
                \draw[->, draw=red] ($(confidence map window) + (.2,0)$) -- ($(confidence map window) + (1,0)$);
                \draw[->, draw=red] ($(confidence map window) + (0,-.2)$) -- ($(confidence map window) + (0,-.7)$);
                \draw[->,draw=blue] (classifier) -- ($(confidence map window)$);
            \end{tikzpicture}
            \caption{Sliding window architecture}\label{fig:sliding-window}
        \end{figure}
        \par{
            This cropped regions may overlap. In those circumstances, an euristic to combine different values of confidences is needed. In this article ** TODO completare... **.
        }
        \par{
            In figure \ref{fig:sliding-window} the sketch of this architecture is shown. In the illustration the resulting map, called \emph{confidence map}, is associated to an array of confidences relative to the possible labels. 
        }
        \par{
            The \emph{confidence map} is then used along with some image segmentation tecnique to spot defective regions. In particular, a $n+1$ levels whatershed algorithm is proposed in \ref{section:challenger:image-segmentation}.
        }
        \par{
            Since defects may have different dimensions, more refined tecniques could be used to improve the accuracy of the challenger. However, these are outside the scope of this work, and they are proposed as a further development in \ref{section:further-work}. Moreover, the segmentation tecnique proposed in \ref{section:challenger:image-segmentation} soothes this problem, since it combines local information from different areas to build the defective regions.
        }
    \subsection{Image segmentation}\label{section:challenger:image-segmentation}
        \par{
            The \emph{confidence map} is used to segmentate the image through a $n+1$ levels whatershed algorithm. However, since the defective regions are always disjunct, the problem can be reduced to a binary whatershed algorithm \cite{ieee:87344} considering all the defective classes as one, and distinguishing them only later. The $n+1$ levels whatershed algorithm is left as a further development in \ref{section:further-work}.
        }
        \par{
            As a final remark about the challenger, it is patent that even this naive implementation is far more involved then the proposed architecture, and the reason lies on the image segmentation approach.
        }