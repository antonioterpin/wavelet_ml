\section{Results}\label{section:results}
    Review the article, make some considerations on results

    \begin{figure}
        \centering
        % \includegraphics[width=.8\linewidth]{graphics/results/mc-cnn-shape-confusion}
        \caption{Shape column confusion matrix on ideal input.}\label{fig:shape-confusion}
    \end{figure}

    \begin{figure}
        \centering
        % \includegraphics[width=.8\linewidth]{graphics/results/mc-cnn-local-confusion}
        \caption{Local column confusion matrix on ideal input.}\label{fig:local-confusion}
    \end{figure}

    \begin{figure}
        \centering
        % \includegraphics[width=.8\linewidth]{graphics/results/mc-cnn-global-confusion}
        \caption{Global column confusion matrix on ideal input.}\label{fig:global-confusion}
    \end{figure}

    \begin{figure}
        \centering
        % \includegraphics[width=.8\linewidth]{graphics/results/mc-cnn-final-classifier-confusion}
        \caption{Final classifier confusion matrix on ideal input.}\label{fig:final-classifier-confusion}
    \end{figure}

    \begin{table}
        \caption{Results}\label{table:results}
    \end{table}

    \par{
        *** Confronto con esplicitamente indicati il contributo di ogni step al miglioramento del risultato *** Results are shown in table \ref{table:results}.
    }

    \par{
        The whole system implementation can be found in the \href{https://github.com/antonioterpin/wavelet_ml}{\texttt{GitHub}} repository \cite{antonioterpin:github}.
    }

\section{Further work}\label{section:further-work}
    \par{
        This work can be further developed in many ways.
    }
    \par{
        Firstly, the challenger could be refined in order to evaluate better the effectiveness of the proposed architecture. As an example, a multi-scale approach could be considered, since the challenger does not take into account the variability of defects dimensions. To solve this flaw, \emph{image pyramids} could be used.
    }
    \par{
        Secondly, other refined architectures could be compared to the proposed one.
    }
    \par{
        Finally, an interesting development would be to investigate further a segmentation with overlapping (adjacent) regions, when there are more classes then foreground/background only. Towards this multi-level segmentation, it would be interesting to consider a non-binary whatershed-based algorithm.
    }