\section{Detector}\label{section:region_proposals}
    \par{
        The second stage of the proposed architecture is a \emph{detector} for region proposals, which are then fed into the \emph{classificator} for classification, which is described in \ref{section:mc-cnn}.
    }
    \par{
        The whole system relies on the \emph{detector} to spot one or more region of interest (ROI), which are the plausible candidates to be defective regions.
    }
    \par{
        To extract them, firstly an edge detector is used (\ref{subsection:contour_detection}). Its output is a binary matrix describing the edges found. The alpha shape of this matrix is then used to segmentate ROIs (\ref{subsection:segmentation}). Finally, the bounding box of the regions can be easily calculated (\ref{subsection:bounding_box}).
    }
    \subsection{Edge detection}\label{subsection:contour_detection}
        \par{
            The first step towards segmentation is .... ****
        }
        \par{
            ** Explain kovesi ecc.. **
        }
        \par{
            ** Hysteretic thresholding **
        }
        \par{
            The parameters *** inserire parametri e loro dominio *** are tuned through Bayesian optimization  \cite{arXiv:2012arXiv1206.2944S,arXiv:2018arXiv180702811F, matlab:bayesian-opt}, together with the parameteres in \ref{subsection:segmentation}.
        }
    \subsection{Image Segmentation}\label{subsection:segmentation}
        \par{
            Given the plausible edges found by the previous step, the image segmentation is done banking on alpha shapes \cite{springer:10.1007/11907350_46}.
        }
        \par{
            The Delaunay triangulation $\mathcal{D}$ od a set of point $\mathcal{S}$ is the subset of all triangles $T = \left\{\left(a, b, c\right) \subset \mathcal{S}^3, a \neq b, b \neq c, a \neq c \right\}$ such that $t \in T, x \in \mathcal{S} \Rightarrow x \not\in \mathcal{C}\left(t\right)$, where $\mathcal{C}\left(t\right)$ is the circumcircle of $t$.
        }
        \par{
            The union of cells $c \in \mathcal{D}$ is called polytope of $\mathcal{D}$. 
            \begin{equation*}
                \mathcal{D}_\alpha = \left\{t \in T \colon r\left(\mathcal{C}\left(t\right)\right) < \alpha\right\}
            \end{equation*}
            Where $r\left(\mathcal{C}\left(t\right)\right)$ is the radius of the circumcircle of $t$. The polytope of $\mathcal{D}_\alpha$ is called \emph{alpha-shape}.
        }
        \begin{figure}
            \centering
            \includegraphics[width=\linewidth]{graphics/architecture/detector-points}
            \includegraphics[width=\linewidth]{graphics/architecture/detector-a-shape}
            \includegraphics[width=\linewidth]{graphics/architecture/detector-a-shape-better-radius}
            \caption{Alpha shape with different alphas of a set of points.}\label{fig:example-alpha-shape}
        \end{figure}
        \begin{figure}
            \centering
            \includegraphics[width=\linewidth]{graphics/architecture/detector-convhull}
            \caption{Convex hull of a set of points.}\label{fig:example-convex-hull}
        \end{figure}
        \par{
            As an example, in figure \ref{fig:example-alpha-shape} are compared the alpha-shapes with different $\alpha$ values of a set of points. In figure \ref{fig:example-convex-hull} is shown the convex-hull of the same set of points. The choice of alpha shapes for the given task is clear.
        }
        \par{
            However, in order to obtain topologically correct image segmentation, three parameters need to be properly choosen \cite{springer:10.1007/11907350_46}. These are the \emph{alpha-radius}, the \emph{hole-thresholding} and the \emph{region-thresholding} \cite{matlab:alpha-shape}.
        }
        \par{
            The first is the value of $\alpha$, the second is the maximum area of interior holes that is filled and the third is the largest area that is suppressed (ignored).
        }
        \par{
            Observe that the latter is usefull to soothe the effects of noisy edge detections.
        }
        \par{
            However, tuning them properly manually is quixotic, hence they are choosen through Bayesian optimization.
        }
        \par{
            The segmentation step outputs a set of pixels, which can be compared with the ideal segmentation, i.e. only the pixels within some defective region.
        }
        \par{
            Therefore, the goodness of fit of the set of proposed pixels $\left(X\right)$ to the ideal pixels $\left(Y\right)$ can be measured as:
            \begin{equation*}
                \chi\left(X, Y\right) = \frac{2 \lvert X \cap Y \rvert}{\lvert X \rvert + \lvert Y \rvert}
            \end{equation*}
        }
        \par{
            Hence, the loss function is:
            \begin{equation*}
                \mathcal{L}\left(X, Y\right) = 1 - \chi\left(X, Y\right)
            \end{equation*}
        }
        \par{
            The acquisition function used is \emph{expected-improvement-plus} \cite{matlab:acquisition}.
        }
        \par{
            As a final consideration, using alpha shape lessen the detrimental effect of the noise in edge detection. Indeed, outliers are suppressed by the hole threshold and region threshold parameteres.
        }
        \begin{figure}
            \caption{Bayesian optimization on image segmentation.}\label{fig:image-segmentation-optimization}
        \end{figure}
        \par{
            In figure \ref{fig:image-segmentation-optimization} ** DESCRIVERE OUTPUT BAYESIAN **
        }
        \par{
            The regions of this alpha shape are the ROIs, which are fed into the \emph{classificator}.
        }
    \subsection{Bounding box}\label{subsection:bounding_box}
        \par{
            The bounding box of a ROI can be easily calculated considering the extremal coordinates of its pixels.
        }
        \par{
            Let \texttt{pixels} be the $M \times 2$ matrix of the coordinates of the pixels within the ROI, with $M$ the number of pixels. Then the bounding box is: 
        }
        \par{
            \begin{BVerbatim}

                 _         _         
                | minX minY |
bounding_box =  | maxX maxY |
                 -         - 
            \end{BVerbatim}
        }
        \par{
            Where:
        }
        \par{
            \begin{BVerbatim} 

        minX = min(pixels(:,1))
        maxX = max(pixels(:,1))
        minY = min(pixels(:,2))
        maxY = max(pixels(:,2))
   
            \end{BVerbatim}
        }



% \subsection{Edge detection filter}
%         Cite main edge detection filters, explain approach (phase congruency) and previous use of wavelet in this field.. our approach and our evaluation of different wavelet families.
%         \subsubsection{Phase congruency edge detection}
%             Phase congruency.....  through wavelet .....
%         \subsubsection{...}
%             ..... Other steps .....
%         \subsubsection{Maxima suppression}
%             Describe maxima suppression....
%         \subsubsection{Thresholding}
%             Describe hysteresys thresholding....
%             Table comparing different thresholding values....
%         \subsubsection{Wavelet family}
%             Present different wavelet families, euristic considerations, ....
%             Table comparing different wavelet families....
%         \subsubsection{Evaluation schema}
%             Describe how we evaluated any choice.. optimization for example on confronting how much the defects area are highlighted.
%             An effective approach to this black-box derivative-free global-optimization method is Bayesian Optimization \cite{rasmussen:williams:2006, arXiv:2018arXiv180702811F, arXiv:2012arXiv1206.2944S}
%             Loss when we loose defects...
%             Loss when we keep too much non defects area...